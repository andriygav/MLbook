\documentclass{article}
\usepackage{graphicx} % Required for inserting images
\usepackage{graphicx} % Required for inserting images
\usepackage{graphicx} % Required for inserting images
\usepackage[a4paper,top=1cm,bottom=1cm,left=2cm,right=2cm]{geometry}
\usepackage[utf8]{inputenc}
\usepackage[russian]{babel}
\usepackage{amsmath}
\title{MLhw2}



\begin{document}

\section*{Нестандартные функции потерь. Функции потерь для обработки временных рядов.}

Временные ряды часто имеют уникальные особенности, такие как сезонность, тренд и автокорреляция, которые необходимо учитывать при построение модели. Поэтому использование стандартных функций потерь не всегда рационально.

Сезонность относится к периодическим колебаниям в данных временных рядов, которые происходят в определенные временные интервалы. Это могут быть ежедневные, еженедельные, ежемесячные или годовые паттерны. Например, увеличение продаж в праздничные периоды.
Поэтому требуются функции потерь, способные учитывать периодические паттерны, важно правильно взвешивать ошибки для разных периодов. Стандартные функции потерь не справляются с этой задачей.

Тренд - долгосрочные изменения в данных, которые могут быть линейными или нелинейными,
Например, рост цен на жилье связанное с ростом населения города или экономики в целом.
Функции потерь должны адекватно работать при наличии долгосрочного тренда, важно учитывать разные масштабы значений в начале и конце прогнозируемого ряда.

Автокорреляция это мера корреляции между значениями ряда с разницей во времени, зависимость текущих значений от предыдущих, что требует учета временной структуры данных.
Стандартные функции потерь не учитывают эту зависимость, тогда как, специальные функции могут учитывать структуру автокорреляции.

Далее используются следующие обозначения:\\
$y_{i,t}$ - значение $i$ временного ряда в момент времени $t$.\\
$f_{i,t}$ - предсказанное значение $i$ временного ряда в момент времени $t$.\\
$f_{i,t}^q$ - предсказанный квантиль $q$ временного ряда $i$ в момент времени $t$\\
$N$ - количество временных рядов.\\
$T$ -  длина временного ряда.\\
$H$ - длина предсказанных значений.\\

Нестандартные функции потерь используемые для работы с временными рядами:\\
\\
$MASE$ (Mean Absolute Scaled Error)

$$MASE = \frac{1}{N} \frac{1}{H} \sum_{i = 1}^{N} \frac{1}{a_{i}}\sum_{t = T + 1} ^ {T + H}|y_{i,t} - f_{i,t}| $$ , где $a_{i}$
$$a_{i} = \frac{1}{T - m} \sum_{t = m + 1}^{T}|y_{i,t} - y_{i, t - m}|$$, для $m$ - период сезонности.

Фактически $MASE$ это $MAE$ предсказанных значений разделенное на $MAE$ наивного прогноза(предсказанное значение это последнее наблюдаемое значение), с поправкой на сезонность ряда.
$MASE$ устойчивее к выбросам, учитывает сезонность поэтому подходят для оценки временных рядов.\\
\\
$SQL$ Scaled quantile loss.
$$SQL = \frac{1}{N} \frac{1}{H} \sum_{i = 1}^{N} \frac{1}{a_{i}}\sum_{t = T + 1} ^ {T + H}p_{q}(y_{i,t}, f_{i,t}^{q})$$, где
$$p_{q}(y_{i,t}, f_{i,t}^{q}) = \begin{cases}
    2(1 - q)(f_{i,t}^q - y_{i,t}), если y_{i,t} < f_{i,t}^q\\
    2q(y_{i,t} - f_{i,t}^q ),если y_{i,t} \geq f_{i,t}^q
\end{cases}$$
\\
Совпадает с $MASE$, если $q = \frac{1}{2}$, устойчивость к выбросам, позволяет оценивать качество предсказаний на разных уровнях (квинтилях), что дает более полное представление о распределении ошибок. Это полезно, когда важно учитывать как нижние, так и верхние границы предсказаний. Это делает $SQL$ полезным инструментом для анализа и предсказания временных рядов, особенно в ситуациях, когда важно учитывать распределение ошибок и устойчивость к выбросам.\\
\\
Функция потерь учитывающая автокорреляцию. ACF loss может быть определена как разница между автокорреляцией фактических значений и автокорреляцией предсказанных значений

Для временного ряда - автокорреляция на лаге k определяется как:
    \[
   \text{ACF}(k) = \frac{\sum_{t=k+1}^{N} (y_t - \bar{y})(y_{t-k} - \bar{y})}{\sum_{t=1}^{N} (y_t - \bar{y})^2}
   \]

   где \( \bar{y} \) — среднее значение временного ряда\\

$ACF_$ loss
   \[
   \text{ACF loss} = \sum_{k=1}^{K} \left( \text{ACF}_{\text{actual}}(k) - \text{ACF}_{\text{predicted}}(k) \right)^2
   \]
   где \( K \) — максимальный лаг, на котором вы хотите оценивать автокорреляцию.
Таким образом, ACF loss измеряет, насколько хорошо предсказанные значения сохраняют автокорреляционные свойства временного ряда. \\
\\
Задача 1\\
\\
При прогнозировании спроса на товар важнее не допустить нехватки товара, чем его переизбытка. Какую функцию потерь следует использовать?
\\
\\
Ответ:
Scaled Quantile Loss с $q$ > 0.5 (например, 0.8), что приведет к более "осторожным" прогнозам с меньшей вероятностью недооценки, также такая фукнция потерь учитывает сезонность, в отличие от стандартных функций потерь.
\\
\\
Задача 2\\
\\
Для задачи финансового анализа необходимо предсказывать цены акций, которые имеют сильные временные зависимости. Какую функцию потерь следует использовать для этой задачи и почему?
\\
\\
Ответ: Используйте ACF loss, чтобы сохранить автокорреляционные свойства временного ряда. Это важно, так как финансовые данные часто имеют зависимость от времени, и использование ACF loss поможет избежать искажений в оценках риска и улучшить качество предсказаний, сохраняя структуру временных зависимостей.
\\
\\
Задача 3\\
Компании необходимо прогнозировать ежемесячные продажи. Товар компании имеет явную сезонность, какую фукнцию потерь лучше использовать и почему?
\\
Ответ: MASE будет оптимальным выбором, так как она, в отличие от стандартных функций, учитывает сезонность и менее чувствительна к выбросам.


\\


\end{document}
