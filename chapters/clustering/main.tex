\section{Критерии качества кластеризации}

Давайте детально разберем основные метрики, используемые для оценки качества кластеризации данных.  

Выбор подходящей метрики напрямую зависит от наличия или отсутствия предварительной разметки данных и от того, задано ли количество кластеров априори или оно подбирается в процессе кластеризации.

\subsection{Критерии, не требующие разметки выборки}

\subsubsection{Среднее внутрикластерное расстояние} \hfill\\

Название этой метрики говорит само за себя: она отражает среднее расстояние между всеми парами точек внутри одного кластера.  Иными словами, мы подсчитываем сумму расстояний между всеми парами точек в каждом кластере и делим на общее количество таких пар.  

Формула метрики выглядит так:
\begin{equation*}
     F_0 = \cfrac{\displaystyle\sum_{i=1}^n \sum_{j=i}^n \rho(x_i,  x_j) I[a(  x_i)=a(x_j)]}{\displaystyle\sum_{i=1}^n \sum _ {j=i}^n I[ a(x_i)=a(x_j)]}.
\end{equation*}

В формуле учитываются и пары вида $(x_i, x_i)$, что позволяет избежать неопределенности $\frac{0}{0}$ в случае, если кластер состоит всего из одной точки.  Однако, иногда для упрощения вычислений суммирование ведется только по парам $(x_i, x_j)$, где $i < j$, при этом для случая одноточечных кластеров значение метрики полагается равным нулю.

Вычисление этого критерия достаточно трудоёмко, поэтому можно также ввести средний квадрат внутрикластерного расстояния, если нам известные представители, или центры масс, кластеров $\mu_k$:
\begin{equation*}
     \Phi_0 = \displaystyle\frac{1}{nK} \sum_{k=1}^K \sum_{i=1}^n \rho(\mu_k,  x_i)^2 I[a(x_i)=k].
\end{equation*}

Наша цель при кластеризации -- получить максимально компактные кластеры, поэтому мы стремимся минимизировать значение этой метрики.  Чем меньше среднее внутрикластерное расстояние, тем лучше.

\subsubsection{Среднее межкластерное расстояние} \hfill\\

В отличие от предыдущей метрики, среднее межкластерное расстояние оценивает среднее расстояние между точками из разных кластеров.  

Формула выглядит следующим образом:
\begin{equation*}
     F_1 = \cfrac{\displaystyle\sum_{i=1}^n \sum_{j=i}^n \rho(x_i,  x_j) I[a(  x_i)\ne a(x_j)]}{\displaystyle\sum_{i=1}^n \sum _ {j=i}^n I[ a(x_i)\ne a(x_j)]}.
\end{equation*}

Здесь, наоборот, мы стремимся к максимизации этого значения.  Чем больше расстояние между кластерами, тем лучше разделение.  

\subsection{Критерии, требующие разметки выборки}

Следующие метрики требуют, чтобы мы заранее знали, к какому классу принадлежит каждый объект в наборе данных.  Это позволяет сравнить результаты кластеризации с истинным распределением данных.

Мы будем обозначать кластеры, полученные в результате кластеризации, как $k \in \{1, \ldots, K\}$, а истинные классы -- как $c \in \{1, \ldots, C\}$.

\subsubsection{Гомогенность} \hfill\\

Если у нас есть разметка, то можно свести задачу кластеризации к использованию методов классификации. Если размеченных данных достаточно много, то обучение классификатора -- более подходящий подход. Однако часто возникает ситуация, когда данных достаточно для оценки качества кластеризации, но всё ещё не хватает для использования методов обучения с учителем.

Пусть $n$ -- общее количество объектов в выборке, $n_k$ -- количество объектов в кластере номер $k$, $m_c$ -- количество объектов в классе номер $c$, а $n_{c,k}$ количество объектов из класса $c$ в кластере $k$. Рассмотрим следующие величины:
\begin{gather*}
    H_{class} = -\displaystyle\sum_{c=1}^C \cfrac{m_c}{n} \log\cfrac{m_c}{n}, \\
    H_{clust} = -\displaystyle\sum_{k=1}^K \cfrac{n_k}{n} \log\cfrac{n_k}{n}, \\
    H_{class \vert clust} = -\displaystyle\sum_{c=1}^C \sum_{k=1}^K \cfrac{n_{c,k}}{n} \log\cfrac{n_{c,k}}{n_k}.
\end{gather*}

Несложно заметить, что эти величины соответствуют формуле энтропии и условной энтропии для мультиномиальных распределений $\cfrac{m_c}{n}, \cfrac{n_k}{n}, \cfrac{n_{c,k}}{n_k}$ соответственно.

Гомогенность кластеризации определяется такой формулой:
\begin{equation*}
    Homogeneity = 1 - \cfrac{H_{class \vert clust}}{H_{class}}.
\end{equation*}

Отношение $\cfrac{H_{class \vert clust}}{H_{class}}$ показывает, насколько уменьшается неопределенность в распределении классов (измеряемая энтропией), если мы знаем, к какому кластеру относится каждый объект. 

Худший сценарий -- когда отношение равно единице, то есть знание о кластерной принадлежности никак не помогает определить истинный класс объекта (энтропия не изменилась). 

Лучший случай -- когда каждый кластер содержит только объекты одного класса, и, следовательно, зная номер кластера, мы точно знаем истинный класс (гомогенность равна 1). Заметим, что тривиальный (и бессмысленный) способ добиться максимальной гомогенности -- это выделить каждый объект в отдельный кластер.

\subsubsection{Полнота} \hfill\\

Метрика полноты аналогична гомогенности, но использует условную энтропию $H_{clust \vert class}$, которая симметрична $H_{class \vert clust}$:
\begin{equation*}
    Completeness = 1 - \cfrac{H_{clust \vert class}}{H_{clust}}.
\end{equation*}

Полнота равна единице, когда все объекты, принадлежащие одному и тому же истинному классу, находятся в одном и том же кластере.

Тривиальный, но непрактичный способ получить максимальную полноту -- это объединить все объекты выборки в один большой кластер.

\subsubsection{V-мера} \hfill\\

Метрики гомогенности и полноты в кластеризации аналогичны точности и полноте в классификации.  V-мера, в свою очередь, аналогична F-мере и представляет собой гармоническое среднее гомогенности и полноты. Пусть $\beta$ - весовой параметр, тогда формула выглядит следующим образом:
\begin{equation*}
    V_\beta = \cfrac{(1+\beta) \cdot Homogeneity \cdot Completeness}{\beta \cdot Homogeneity + Completeness}.
\end{equation*}

В случае $\beta = 1$ получаем, что $V_1$-мера является просто средним гармоническим гомогенности и полноты:
\begin{equation*}
    V_\beta = \cfrac{2 \cdot Homogeneity \cdot Completeness}{Homogeneity + Completeness}.
\end{equation*}

Использование V-меры позволяет избежать тривиальных решений, таких как присвоение каждого объекта в отдельный кластер (максимальная гомогенность) или объединение всех объектов в один кластер (максимальная полнота).  V-мера обеспечивает сбалансированную оценку качества кластеризации, учитывая как гомогенность, так и полноту.

\subsubsection{Коэффициент силуэта} \hfill\\

Коэффициент силуэта — метрика качества кластеризации, не требующая наличия меток классов. Сначала он вычисляется для каждого объекта, а затем итоговая метрика для всей выборки определяется как среднее значение коэффициентов силуэта всех объектов.

Для вычисления коэффициента силуэта $S(x_i)$ нужны две величины:

\begin{itemize}
    \item $A(x_i)$ — среднее расстояние от объекта $x_i$ до всех других объектов в том же кластере.
    \item $B(x_i)$ — среднее расстояние от объекта $x_i$ до объектов ближайшего соседнего кластера.
\end{itemize}

Сам коэффициент силуэта вычисляется по формуле:
\begin{equation*}
    S(x_i) = \cfrac{B(x_i)-A(x_i)}{\max (B(x_i), A(x_i))}.
\end{equation*}

В идеале, точки внутри кластера должны быть ближе друг к другу, чем к точкам ближайшего соседнего кластера $A(x_i) < B(x_i)$. Однако это не всегда так.  Например, если кластер сильно вытянут или большой, а рядом находится небольшой кластер, то среднее расстояние до точек своего кластера ($A(x_i)$) может оказаться больше, чем до точек соседнего ($B(x_i)$).  Поэтому разность $B(x_i) - A(x_i)$ может быть как положительной, так и отрицательной, хотя в идеале ожидается положительное значение.  Коэффициент силуэта $S(x_i)$, изменяющийся от -1 до +1, максимизируется, когда кластеры компактны и хорошо разделены.

Главное преимущество коэффициента силуэта — он не требует меток классов и позволяет оценивать качество кластеризации при разных количествах кластеров.

\subsection{Различия и выбор метрик качества кластеризации}

Выбор метрики качества кластеризации зависит от нескольких факторов. Если число кластеров известно и разметка данных отсутствует, то целесообразно использовать среднее внутрикластерное или среднее межкластерное расстояние для оптимизации качества кластеризации. 

Если же доступна разметка данных, то для оценки качества можно использовать гомогенность и полноту, а V-мера, сочетающая эти две метрики, позволяет также подбирать оптимальное число кластеров.

В случае отсутствия разметки и неизвестного числа кластеров, коэффициент силуэта является наиболее подходящей метрикой на практике. Исключение составляет ситуация, когда результаты кластеризации используются как промежуточный этап в более сложной задаче. В таких случаях качество кластеризации оценивается косвенно, по качеству решения конечной задачи, и выбор алгоритма кластеризации и его параметров подчиняется этой цели.

\subsection{Задачи на понимание}
\subsubsection{Задача 1}

Представьте, что у вас есть два набора данных с одинаковым средним внутрикластерным расстоянием. Может ли это означать, что качество кластеризации в обоих наборах одинаково? Объясните, почему да или нет.

\subsubsection{Ответ}

Нет, одинаковое среднее внутрикластерное расстояние не гарантирует одинаковое качество кластеризации. Эта метрика отражает только компактность кластеров, игнорируя другие важные аспекты, такие как разделение между кластерами, форма кластеров и наличие выбросов. Например, в одном наборе данных кластеры могут быть компактными и хорошо разделены, а в другом -- компактными, но сильно перекрывающимися. Среднее внутрикластерное расстояние будет одинаковым, но качество кластеризации -- разным.

\subsubsection{Задача 2}

У вас есть данные, где границы между кластерами размыты, и некоторые точки могут принадлежать нескольким кластерам одновременно. Какая метрика качества кластеризации наименее подходит для оценки результатов в этом случае, и почему?

\subsubsection{Ответ}

Метрики, основанные на жестком распределении точек по кластерам (например, среднее внутрикластерное расстояние, среднее межкластерное расстояние), наименее подходят. Это связано с тем, что они предполагают, что каждая точка строго принадлежит только одному кластеру. В случае нечетких кластеров более подходящими могут быть метрики, учитывающие степень принадлежности точки к каждому кластеру.

\subsubsection{Задача 3}

Опишите сценарий, в котором высокая гомогенность, но низкая полнота. И наоборот.

\subsubsection{Ответ}

Высокая гомогенность, низкая полнота. Представим, что у нас есть два истинных класса A и B. Алгоритм кластеризации создал много маленьких кластеров, каждый из которых содержит преимущественно объекты одного класса (высокая гомогенность). Однако объекты одного и того же класса (например, класса A) разбросаны по множеству разных кластеров. В этом случае полнота будет низкой, так как объекты одного класса не собраны вместе.

Высокая полнота, низкая гомогенность. В этом случае объекты одного класса сгруппированы в одном кластере (высокая полнота). Однако этот кластер содержит значительное количество объектов из других классов, что приводит к низкой гомогенности. Например, один большой кластер может содержать значительное количество объектов класса A и меньшее -- класса B. Полнота для класса A высокая, а гомогенность низкая, потому что кластер "загрязнен" объектами класса B.

\section{DBSCAN}

\textbf{DBSCAN} (Density-Based Spatial Clustering of Applications with Noise) - алгоритм кластеризации, решающий проблему сО сферичностью кластеров, он не делает никаких предположений о форме кластеров. Также он довольно быстрый и подходит для кластеризации больших данных.
\\
Он основан на понятии {\textit{окрестности}}.

\textbf{Определение 1.} Задан объект $x \in U$, его $\varepsilon$-окрестность $U_\varepsilon (x) = \{\;u\in U:\; \rho (x,u) \leq \varepsilon \;\}$ - это множество объектов, которые находятся на расстоянии не больше $\varepsilon$ от заданного объекта $x$.

Тогда каждый объект может быть отнесен к одному из трёх типов:
\begin{itemize}
    \item \textit{корневой}: имеющий плотную окрестность,  {$\abs{U_\varepsilon (x)} \geq m$}, т.е. $\varepsilon$ содержит $\geq m$ объектов.
    \item \textit{граничный}: не корневой, но в окрестности корневого.
    \item \textit{шумовой (выброс)}: не корневой и не граничный.
\end{itemize}
\begin{figure}[h!]
    \centering
    \includegraphics[width=0.9\linewidth]{png/An-Example-Illustrating-the-Density-Based-DBSCAN-Clustering-Method-Applied-to-SMLM-Data.png}
    \caption{An Example Illustrating the Density-Based DBSCAN Clustering Method Applied to SMLM Data}
    \label{fig:enter-label}
\end{figure}
Возникает 2 параметра: $\varepsilon$ и $m$. Других параметров не будет. От этих параметров и будет зависеть то, какой картина кластеризации получится. Также к преимуществам этого метода относится то, что он не задает заранее количество кластеров, в отличие, например, от k-means, причём количество кластеров будет зависеть от $\varepsilon$ и $m$. 

Как работает алгоритм: берётся произвольная точка, если она имеет плотную окрестность, то дальше рассматривается каждая точка этой плотной окрестности, и вокруг неё также строится $\varepsilon$-окрестность, и так пока не будет достигнута граница некоторого множества объектов. 

Хорошей аналогией может служить лес: один лес - это один кластер, через опушку, второй лес, - другой кластер, мы находимся в лесу. Смотрим, в нашей окрестности деревьев много, это значит, что мы в корневой точке находимся, и дальше мы идём, пока не выйдем на опушку леса, там мы окажемся в граничной точке - она уже не корневая, вокруг деревьев меньше. А где-то могут расти отдельно стоящие деревья - это шумовые выбросы. И вот так ходим по лесу, пока его весь не обойдём, и как только мы обошли весь лес, назовем его кластером. После чего случайно выбираем новое дерево и начинаем строить другой кластер.

Формализуем алгоритм в виде псевдокода:\\
\begin{tabularx}{\linewidth}{lX}
\textbf{вход:} выборка $X^l - \{x_1,...,x_l\}$; параметры $\varepsilon$ и $m$\\
\textbf{выход:} разбиение выборки на кластеры и шумовые выбросы;\\\hspace*{7mm}\hspace*{9mm}$U := X^l$ - не помеченные точки, $a := 0$\\
\textbf{пока} в выборке есть непомеченные точки, $U \neq \emptyset$:\\
\hspace*{7mm} взять случайную точку $x \in U$; \\
\hspace*{7mm} \textbf{если} $\abs{U_\varepsilon (x)} < m$ \textbf{то} \\
\hspace*{7mm}\hspace*{7mm} пометить $x$ как, возможно, шумовой;\\
\hspace*{7mm}\textbf{иначе} \\
\hspace*{7mm}\hspace*{7mm} создать новый кластер: $K:=U_\varepsilon (x); \; a:=a+1;$ \\
\hspace*{7mm}\hspace*{7mm} \textbf{для всех} $x' \in K$, не помеченных или шумовых \\
\hspace*{7mm}\hspace*{7mm}\hspace*{7mm} \textbf{если} $\abs{U_\varepsilon (x')} \geq m$,  \textbf{то} $K := K \cup U_\varepsilon (x')$; \\
\hspace*{7mm}\hspace*{7mm}\hspace*{7mm} \textbf{иначе} поментить $x'$ как граничный кластера $K$;\\
\hspace*{7mm}\hspace*{7mm} $a_j := a$ для всех $x_i \in K$;\\
\hspace*{7mm}\hspace*{7mm} $U := U \textbackslash K$;\\
\vspace{5mm}
\end{tabularx}

В таком виде алгоритм обладает следующими \textbf{свойствами}:
\begin{itemize}
    \item быстрая кластеризация больших данных: \\$O(l^2)$ в худшем случае, \\ $O(l \mathrm{ln} l)$ при эффективной реализации $U_\varepsilon (x)$;
    \item кластеры произвольной формы
    \item деление объектов на корневые, граничные, шумовые.
\end{itemize}

При этом важно понимать, что граничные объекты не выстраивают в точности границу каждого кластера. Практически это означает, что не стоит всерьез рассматривать граничные объекты, в отличие от шумовых, которые действительно можно в дальнейшем анализировать.

\subsection{Примечание о HDBSCAN} 
От гиперпараметра $\varepsilon$ можно избавиться, используя дивизивную кластеризацию. Такая модификация называется HDBSCAN. Его суть проста: необходимо построить дендрограмму, где по $Оу$ будет отложен $\varepsilon$ (на рис.\ref{fig:hdbdendro} снизу distance). Так мы сможем явно видеть вложенные кластеры. Алгоритм затем сам вычисляет оптимальное количество кластеров на основе метрики "стабильности кластеров".

\begin{figure}[h!]
    \centering
    \includegraphics[width=0.6\linewidth]{png/hdbscan_dendrogramm.png}
    \caption{К примечанию о HDBSCAN}
    \label{fig:hdbdendro}
\end{figure}
\subsection{Задачи}
\textbf{Задача 1.}

\textbf{Условие.} Применить DBSCAN для выборки из таблицы с $m=4,\;\varepsilon=1.9$. Метрика евклидова.

\begin{center}
\begin{tabular}{ |c|c|c| } 
 \hline
 P1(3,7) & P5(7,3) & P9(3,3) \\ 
 P2(4,6) & P6(6,2) & P10(2,6) \\ 
 P3(5,5) & P7(7,2) & P11(3,5) \\ 
 P4(6,4) & P8(8,4) & P12(2,4) \\ 
 \hline
\end{tabular}
\end{center}

\textbf{Решение.}
Запишем матрицу, составленную из соответственных расстояний между точками выборки:
\begin{center}
\begin{tabular}{ |c|c|c|c|c|c|c|c|c|c|c|c|c|} 
 \hline
dot & P1 & P2 & P3 & P4 & P5 & P6 & P7 & P8 & P9 & P10 & P11 & P12 \\ \hline
P1 & 0 &  &  &  &  &  &  &  &  &  &  &   \\ \hline
P2 & 1.41 & 0 &  &  &  &  &  &  &  &  &  &   \\ \hline
P3 & 2.83 & 1.41 & 0 &  &  &  &  &  &  &  &  &   \\ \hline
P4 & 4.24 & 2.83 & 1.41 & 0 &  &  &  &  &  &  &  &   \\ \hline
P5 & 5.66 & 4.24 & 2.83 & 1.41 & 0 &  &  &  &  &  &  &   \\ \hline
P6 & 5.83 & 4.47 & 3.16 & 2.00 & 1.41 & 0 &  &  &  &  &  &   \\ \hline
P7 & 6.40 & 5.00 & 3.61 & 2.24 & 1.00 & 1.00 & 0 &  &  &  &  &   \\ \hline
P8 & 5.83 & 4.47 & 3.16 & 2.00 & 1.41 & 2.83 & 2.24 & 0 &  &  &  &   \\ \hline
P9 & 4.00 & 3.16 & 2.83 & 3.16 & 4.00 & 3.16 & 4.12 & 5.10 & 0 &  &  &   \\ \hline
P10& 1.41 & 2.00 & 3.16 & 4.47 & 5.83 & 5.83 & 5.66 & 6.40 & 6.32 & 0 &  &   \\ \hline
P11& 2.00 & 1.41 & 2.00 & 3.16 & 4.47 & 4.24 & 5.00 & 5.10 & 2.00 & 1.41 & 0 &   \\ \hline
P12& 2.83 & 3.16 & 4.00 & 5.10 & 4.47 & 5.39 & 6.00 & 1.41 & 2.00 & 2.00 & 1.41 & 0  \\ \hline
\end{tabular}
\end{center}
Сравнивая значения в каждом столбце матрицы с $\varepsilon$ и отбирая те, что меньше этого значения, находим окрестности каждой точки.

\begin{center}
\begin{tabular}{ |c|c| } 
 \hline
 точка & окрестность \\\hline
 P1 & P2, P10\\ 
 P2 & P1, P3, P11\\ 
 P3 & P2, P4\\ 
 P4 & P3, P5\\
 P5 & P4, P6, P7, P8\\
 P6 & P5, P7\\
 P7 & P5, P6\\
 P8 & P5\\
 P9 & P12\\
 P10 & P1, P11\\
 P11 & P2, P10, P12\\
 P12 & P9, P11\\
 \hline
\end{tabular}
\end{center}

Если в окрестности больше $m=4$ точек (включая ее саму), то отнесем эту точку к корневой, иначе - к шумовой.

\begin{center}
\begin{tabular}{ |c|c| } 
 \hline
 точка & тип \\\hline
 P1 & шум\\ 
 P2 & корневая\\ 
 P3 & шум\\ 
 P4 & шум\\
 P5 & корневая\\
 P6 & шум\\
 P7 & шум\\
 P8 & шум\\
 P9 & шум\\
 P10 & шум\\
 P11 & корневая\\
 P12 & шум\\
 \hline
\end{tabular}
\end{center}

Уточним классификацию, учтя граничные точки, т.е. точки, лежащие в окрестности корневых, но при этом не являющимися корневыми:
\begin{center}
\begin{tabular}{ |c|c| } 
 \hline
 точка & тип \\\hline
 P1 & граничная\\ 
 P2 & корневая\\ 
 P3 & граничная\\ 
 P4 & граничная\\
 P5 & корневая\\
 P6 & граничная\\
 P7 & граничная\\
 P8 & граничная\\
 P9 & шум\\
 P10 & граничная\\
 P11 & корневая\\
 P12 & граничная\\
 \hline
\end{tabular}
\end{center}

К первому кластеру отнесем окрестность корневой точки 2, причем в ее окрестности находится еще одна корневая точка 11, так что отнесем и ее окрестность к первому кластеру. Ко второму кластеру отнесем корневую точку 5 и ее окрестность. Осталась лишь одна точка P9, которая не относится ни к какому кластеру и является шумовой.
\begin{figure}[h!]
    \centering
    \includegraphics[width=0.7\linewidth]{png/task1dbs_plot.png}
    \caption{Кластеризация в задаче 1}
    \label{fig:task1dbs}
\end{figure}

\begin{minipage}{.5\textwidth}
\textbf{Задача 2.}\\
\textbf{Условие.}
  Сравните результаты кластеризации с помощью k-means и с помощью DBSCAN и объясните их.\\
\textbf{Решение.}
Объяснение различий:
\begin{itemize}
\item \textit{Форма кластера}:
K-средние: стремится найти сферические или выпуклые кластеры. Предполагается, что кластеры изотропны (однородны во всех направлениях) и имеют схожий размер.
DBSCAN: может обнаруживать кластеры произвольной формы и размера. Не делает предположений о форме кластеров.
\item \textit{Обработка шума}:
K-средние: плохо справляется с шумом. Точки шума могут быть назначены кластерам, что может повлиять на центры кластеров.
DBSCAN: может идентифицировать и маркировать точки шума, которые не назначены ни одному кластеру.
\end{itemize}
\end{minipage}% This must go next to `\end{minipage}`
\begin{minipage}{.4\textwidth}
      \includegraphics[width=0.95\linewidth]{png/task2dbs_plot.png}
\end{minipage}
\begin{itemize}
\item \textit{Плотность кластера}:
K-средние: не учитывает плотность точек. Каждый кластер представлен центроидом.
DBSCAN: учитывает плотность точек. Кластеры формируются на основе плотности точек в окрестности.
\item \textit{Чувствительность параметров}:
K-средние: требует предварительного указания количества кластеров (K), так что, если если заранее указать 3 кластера, то алгоритм и найдет три кластера, даже если он всего один, как на последней паре картинок.
\end{itemize}

\textbf{Задача 3.}\\
\textbf{Предисловие.}
При решении задачи 1 использовалась матрица, состоящая из расстояний между парами точек (\textit{матрица смежности}). Понятием, противоположным расстоянию, является понятие сходства между объектами. Неотрицательная вещественная функция $S(x_i,x_j) = S_{ij}$ называется \textit{мерой сходства}, если:
\begin{itemize}
    \item $0 \leq S(x_i,x_j) < 1$, для $x_i \neq x_j$
    \item $S(x_i,x_j)=1$
    \item $S(x_i,x_j)=S(x_j,x_i)$
\end{itemize}
Пары значений мер сходства можно объединить в \textit{матрицу сходства} $S$, симметричную и единичной диагональю.
\textbf{Условие.}
Применить DBSCAN с пороговым значением \textit{меры сходства} 0.8 и $m = 2$ и заданной матрицей сходства между точками выборки:

\begin{center}
\begin{tabular}{ |c|c|c|c|c|c|} 
 \hline
dot & P1 & P2 & P3 & P4 & P5  \\ \hline
P1 & 1.0 &  &  &  &     \\ \hline
P2 & 0.10 & 1.0 &  &  &  \\ \hline
P3 & 0.41 & 0.64& 1.0 &  & \\ \hline
P4 & 0.55 & 0.47 & 0.44 & 1.0 & \\ \hline
P5 & 0.35 & 0.98 & 0.85 & 0.76 & 1.0 \\ \hline
\end{tabular}
\end{center}

Сравнивая значения в каждом столбце матрицы с $\varepsilon$ и выбирая те точки, для которых значение сходства выше, чем порог, формируем окрестности всех точек.

\begin{center}
\begin{tabular}{ |c|c| } 
 \hline
 точка & окрестность \\\hline
 P1 & -\\ 
 P2 & P5\\ 
 P3 & P5\\ 
 P4 & -\\
 P5 & P2, P3\\
 \hline
\end{tabular}
\end{center}

Если в окрестности больше $m=2$ точек (включая ее саму), то отнесем эту точку к корневой, иначе - к шумовой.

\begin{center}
\begin{tabular}{ |c|c| } 
 \hline
 точка & тип \\\hline
 P1 & шум\\ 
 P2 & корневая\\ 
 P3 & корневая\\ 
 P4 & шум\\
 P5 & корневая\\
 \hline
\end{tabular}
\end{center}

Уточнение классификации, путем учитывания граничных точек, т.е. точек, лежащие в окрестности корневых, но при этом не являющимися корневыми, ничего не дает, т.к. в окрестности точек, определенных как шумовые вообще нет других точек, так что они действительно являются шумом.

К первому кластеру отнесем окрестность корневой точки P2, причем в ее окрестности находятся еще краевая точка P5, так что отнесем ее к этому же кластеру. В окрестности точки P5 помимо уже классифицированной P2 находится еще корневая точка P3, которую также отнесем к первому кластеру. Остальные точки классифицированы как шумовые. Таким образом в данной задаче всего один кластер, состоящий из точек P2, P3, P5.

\section{Алгоритм FOREL}
Рассмотрим алгоритм кластеризации FOREL.
\section*{Необходимые условия работы}

\begin{itemize}
	\item Выполнение гипотезы компактности, предполагающей, что близкие друг к другу объекты с большой вероятностью принадлежат к одному кластеру (таксону).
	\item Наличие линейного или метрического пространства кластеризуемых объектов.
\end{itemize}

\section*{Входные данные}

\begin{enumerate}
	\item Кластеризуемая выборка.
	\begin{itemize}
		\item Может быть задана признаковыми описаниями объектов (линейное пространство) либо матрицей попарных расстояний между объектами.
		\item Замечание: в реальных задачах зачастую хранение всех данных невозможно или бессмыслено, поэтому необходимые данные собираются в процессе кластеризации.
	\end{itemize}
	\item Параметр \( R \) — радиус поиска локальных сгущений.
	\begin{itemize}
		\item Его можно задавать как из априорных соображений (знание о диаметре кластеров), так и настраивать скользящим контролем.
		\item В модификациях возможно введение параметра \( k \) — количества кластеров.
	\end{itemize}
\end{enumerate}

\section*{Выходные данные}

Кластеризация на заранее неизвестное число таксонов.

\section*{Принцип работы}

На каждом шаге мы случайным образом выбираем объект из выборки, раздуваем вокруг него сферу радиуса \( R \), внутри этой сферы выбираем центр тяжести и делаем его центром новой сферы. Таким образом, на каждом шаге мы двигаем сферу в сторону локального сгущения объектов выборки, т.е. стараемся захватить как можно больше объектов выборки сферой фиксированного радиуса. После того как центр сферы стабилизируется, все объекты внутри сферы с этим центром помечаем как кластеризованные и выкидываем их из выборки. Этот процесс повторяется до тех пор, пока вся выборка не будет кластеризована.

\section*{Алгоритм}

\begin{enumerate}
	\item Случайным образом выбираем текущий объект из выборки.
	\item Помечаем объекты выборки, находящиеся на расстоянии менее, чем \( R \) от текущего.
	\item Вычисляем их центр тяжести, помечаем этот центр как новый текущий объект.
	\item Повторяем шаги 2-3, пока новый текущий объект не совпадет с прежним.
	\item Помечаем объекты внутри сферы радиуса \( R \) вокруг текущего объекта как кластеризованные, выкидываем их из выборки.
	\item Повторяем шаги 1-5, пока не будет кластеризована вся выборка.
\end{enumerate}

\subsection{Задачи}

\textbf{Задача 1: Основы работы алгоритма}

\textbf{Условие задачи:}
Какие основные принципы лежат в основе работы алгоритма кластеризации, описанного в тексте? Как алгоритм решает, когда завершить кластеризацию?

\textbf{Решение:}
\begin{itemize}
	\item Центр нового кластера определяется как центр тяжести объектов, находящихся в сфере радиуса \( R \) вокруг выбранного объекта. Алгоритм постепенно двигает сферу в сторону локального сгущения объектов, пока центр тяжести не стабилизируется.
	\item Алгоритм завершает кластеризацию, когда все объекты в выборке оказываются кластеризованными. На каждом шаге выбирается новый объект, формируется его кластер, и эти объекты удаляются из выборки. Процесс повторяется до тех пор, пока не останется необработанных объектов.
\end{itemize}

\vspace{1em}

\textbf{Задача 2: Выбор радиуса}

\textbf{Условие задачи:}
В алгоритме используется параметр \( R \) — радиус поиска локальных сгущений. Как изменение параметра \( R \) повлияет на кластеризацию, если \( R \) слишком или слишком велик?

\textbf{Решение:}
\begin{enumerate}
	\item Если \( R \) слишком мал, то алгоритм будет искать только очень близкие объекты, что может привести к образованию множества маленьких кластеров, каждый из которых содержит лишь несколько объектов. Это приведет к излишней детализированности кластеризации.
	\item Если \( R \) слишком велик, алгоритм будет захватывать слишком много объектов в один кластер, что может привести к неадекватной кластеризации, где несколько разных кластеров окажутся объединены в один.
\end{enumerate}

\vspace{1em}

\textbf{Задача 3: Число кластеров}

\textbf{Условие задачи:}
Какое влияние на эффективность работы алгоритма может оказать использование параметра \( k \) (количества кластеров)? Ответьте на вопросы:
\begin{enumerate}
	\item Что произойдет, если параметр \( k \) установлен заранее и не соответствует истинному числу кластеров в выборке?
	\item Как алгоритм может адаптироваться, если параметр \( k \) не задан и алгоритм должен кластеризовать данные до тех пор, пока не останется необработанных объектов?
\end{enumerate}

\textbf{Решение:}
\begin{enumerate}
	\item Если параметр \( k \) установлен заранее, но не соответствует истинному числу кластеров, то кластеризация будет либо слишком грубой, либо слишком детализированной. Алгоритм может либо объединить несколько кластеров в один, либо разделить один кластер на несколько, что приведет к неадекватной кластеризации.
	\item Если параметр \( k \) не задан, алгоритм будет продолжать кластеризацию до тех пор, пока не обработает все объекты. В этом случае количество кластеров не ограничено, и алгоритм сам находит подходящее число кластеров, основываясь на локальных сгущениях объектов. Этот подход может быть более гибким, но требует большего времени на обработку.